%%%%%%%%%%%%%%%%%%%%%%%%%%%%%%%%%%%%%%%%%
% Beamer Presentation
% LaTeX Template
%
% This template has been downloaded from:
% http://www.LaTeXTemplates.com
%
% It has been edited for teaching purposes by Daniel Krähmer and Dr. Gerrit Bauer (LMU). 
%
% License:
% CC BY-NC-SA 3.0 (http://creativecommons.org/licenses/by-nc-sa/3.0/)
%
%%%%%%%%%%%%%%%%%%%%%%%%%%%%%%%%%%%%%%%%%

%----------------------------------------------------------------------------------------
%	THEMES ET PACKAGES
%----------------------------------------------------------------------------------------

\documentclass{beamer} % Cette commande indique à LaTeX que le document actuel est une présentation (« beamer »).

\usetheme{Boadilla} % Le thème « Boadilla » combine un design simple avec des fonctionnalités utiles. Il en va de même pour des thèmes tels que « Madrid » ou « Pittsburgh ». 

% Voici une série de paquets utiles dont nous aurons besoin ci-après :  
\usepackage[nfrench]{babel} % Définit la langue source de votre document. Cela concerne notamment le format de la date, les guillemets ou la désignation de certaines sections du document (par exemple « Inhaltsverzeichnis » au lieu de « Contents », etc.). Pour les documents en anglais, vous pouvez omettre la commande ou définir \usepackage[english]{babel}.

\usepackage[utf8]{inputenc} % Permet l'utilisation sans problème des caractères spéciaux (par exemple, les trémas, les accents, etc.) ainsi que la césure correcte des mots correspondants.

\usepackage{tikz}

\usepackage{hyperref} % Permet de créer des liens cliquables (par exemple, adresses e-mail, URL, liens dans le document). Tous les détails peuvent être spécifiés via \hypersetup{...}, par exemple \hypersetup{urlcolor=blue} pour des liens URL bleus. x

%------------------------------------------------------------
% Barre de navigation
%------------------------------------------------------------

% Le code suivant permet d'ajouter à notre présentation un pied de page dans lequel tous les chapitres apparaissent sous forme de liens cliquables.
\setbeamertemplate{footline}
{
  \leavevmode%
  \hbox{%
  
  % BLOC 1 : TON NOM (Même style que le numéro de page)
  \begin{beamercolorbox}[wd=.20\paperwidth,ht=2.25ex,dp=1ex,center]{date in head/foot}%
    \usebeamerfont{date in head/foot}%
    Charlotte Amédée
  \end{beamercolorbox}%
  
  % BLOC 2 : NAVIGATION (Couleur contrastée au centre)
  \begin{beamercolorbox}[wd=.70\paperwidth,ht=2.25ex,dp=1ex,center]{author in head/foot}%
    \usebeamerfont{author in head/foot}%
    \insertsectionnavigationhorizontal{.65\paperwidth}{}{}%
  \end{beamercolorbox}%
  
  % BLOC 3 : NUMÉRO DE PAGE
  \begin{beamercolorbox}[wd=.1\paperwidth,ht=2.25ex,dp=1ex,right]{date in head/foot}%
    \usebeamerfont{date in head/foot}
    \insertframenumber{} / \inserttotalframenumber\hspace*{2ex} 
  \end{beamercolorbox}}%
  \vskip0pt%
}
% Supprimer les icônes de navigation
\beamertemplatenavigationsymbolsempty

\usepackage[sfdefault]{libertine}
\usepackage[T1]{fontenc}

%------------------------------------------------------------------
%   Configuration des listes (Gemini) 
%------------------------------------------------------------------
% Couleur des puces de liste (niveau 1, 2 et 3)
\setbeamercolor{itemize item}{fg=TrelloBlue}
\setbeamercolor{itemize subitem}{fg=TrelloBlue}
\setbeamercolor{itemize subsubitem}{fg=TrelloBlue}

%------------------------------------------------------------------
%   COULEURS
%------------------------------------------------------------------
\definecolor{TrelloBlue}{RGB}{17, 104, 227}
\definecolor{Black}{RGB}{35,35,35}
\definecolor{White}{RGB}{255,255,255}
\definecolor{Darkgrey}{RGB}{98,100,104}
\definecolor{Middlegrey}{RGB}{192,193,195}
\definecolor{Lightgrey}{RGB}{230,230,231}
\definecolor{Superlightgrey}{RGB}{245,245,245}

\setbeamercolor{item projected}{bg=TrelloBlue,fg=White}

\setbeamercolor{section in toc}{fg=Darkgrey,bg=White}
\setbeamercolor{alerted text}{fg=TrelloBlue}
\setbeamercolor*{palette primary}{fg=TrelloBlue,bg=Lightgrey}
\setbeamercolor*{palette secondary}{fg=TrelloBlue,bg=Lightgrey}
\setbeamercolor*{palette tertiary}{bg=TrelloBlue,fg=White}
\setbeamercolor*{palette quaternary}{fg=TrelloBlue,bg=Black}

\setbeamercolor*{sidebar}{fg=TrelloBlue,bg=Lightgrey}

\setbeamercolor*{palette sidebar primary}{fg=TrelloBlue}
\setbeamercolor*{palette sidebar secondary}{fg=White}
\setbeamercolor*{palette sidebar tertiary}{fg=TrelloBlue}
\setbeamercolor*{palette sidebar quaternary}{fg=Lightgrey}

\setbeamercolor{titlelike}{parent=palette primary,fg=TrelloBlue}
\setbeamercolor{frametitle}{bg=Lightgrey, fg=TrelloBlue}
\setbeamercolor{frametitle right}{bg=Middlegrey}

\setbeamercolor{separation line}{fg=Darkgrey}
\setbeamercolor{fine separation line}{fg=Darkgrey}

\setbeamercolor*{bibliography entry author}{fg=TrelloBlue}
\setbeamercolor*{bibliography entry note}{fg=TrelloBlue!80!white}

%----------------------------------------------------------------------------------------
%	TITRES DES DIAPOSITIVES 
%----------------------------------------------------------------------------------------
% Vous trouverez ci-dessous des informations LaTeX qui seront utilisées ultérieurement pour créer automatiquement une diapositive de titre. 
% TITRE: 
% \NormalSize = Très grand, \textbf = Gras
\title{\Huge {Outil numérique - Trello}}

% NOM:
\author{\large \textsc{Charlotte Amédée}}

\institute{
    \normalsize{POL 6078 - Outils numériques en sciences sociales} \\ \vspace{0.2cm} % vspace ajoute un petit espace vertical
    \small Étudiante à l'École supérieure d'études internationales \\
    \textit{Université Laval} % \textit = Italique
} 

% DATE:
% \footnotesize = Assez petit
\date{\footnotesize 4 Décembre 2025} 
    
\begin{document} 

\begin{frame} 
\titlepage 
\begin{tikzpicture}[remember picture, overlay]
        \node[anchor=south west, xshift=0.2cm, yshift=0.4cm] 
        at (current page.south west) {
            \includegraphics[width=1.5cm]{Images/logo_ecole.png}
        };
    \end{tikzpicture}
    
\end{frame}


%----------------------------------------------------------------------------------------
%	HISTORIQUE
%----------------------------------------------------------------------------------------
\section{Historique} % La commande \section divise votre document LaTeX en sections. Vous devez donner à ces sections des noms significatifs, car ils apparaîtront dans votre plan.

%\subsection{Subsection Example} %La commande \subsection vous offre un deuxième niveau de structure, analogue à \section

\begin{frame}{Historique}
    \centering
    \includegraphics[height=1cm]{Images/logo_trello.png}
    \vspace{1cm}
    \begin{itemize}
        \setlength\itemsep{1.5em}
        \item Lancement en 2011 par Fog Creek Software
        \item Vendu à Atlassian en 2017
        \item Outil de gestion de projet qui utilise la méthode KANBAN
    \end{itemize}

\begin{tikzpicture}[remember picture, overlay]
        \node[anchor=south west, xshift=0.2cm, yshift=0.4cm] 
        at (current page.south west) {
            \includegraphics[width=1.5cm]{Images/logo_ecole.png}
        };
    \end{tikzpicture}
\end{frame}

%----------------------------------------------------------------------------------------
%	CARACTÉRISTIQUES DE L'OUTIL
%----------------------------------------------------------------------------------------
\section{Caractéristiques}

\begin{frame}[fragile]{Caractéristiques} 
% Important : pour que LaTeX sache ce que singifie une clé bib, vous devez d'abord créer une bibliothèque externe. Celle-ci est intégrée dans ce fichier en haut des paramètres. Vous y enregistrez toutes les informations pertinentes relatives à une publication (par exemple, le titre, les auteurs, l'année de publication, etc.) et définissez une clé Bib . Pour citer ensuite une publication dans votre document LaTeX, vous pouvez toujours utiliser la clé Bib dans la commande cite. 
    \vspace{-0.6cm}
    \begin{columns}[c] % T = aligner le haut des colonnes (Top)
        
        % --- COLONNE GAUCHE (TEXTE) ---
        \begin{column}{0.5\textwidth} % Occupe 50% de la largeur
            \begin{table}
            \small
    \centering
    % Cette commande augmente l'espace vertical entre les lignes 
    \renewcommand{\arraystretch}{2.0} 

    \begin{tabular}{p{0.35\textwidth} | p{0.50\textwidth}}
    
        \textbf{\textcolor{TrelloBlue}{Critère}} & \textbf{\textcolor{TrelloBlue}{Détails}} \\ 
        \hline
            \textbf{Accessibilité} & Modèle Freemium \\
            \textbf{Communauté} & 50M d'utilisateurs \\
            \textbf{Popularité} & Référence de la méthode Kanban \\
            \textbf{Compatibilité} & « Power-Ups » \\
            \textbf{Transparence} & Logiciel propriétaire \\
            \textbf{Adaptabilité} & Totale dans le modèle payant \\
    
        
    \end{tabular}
\end{table}
  \begin{tikzpicture}[remember picture, overlay]
        \node[anchor=south west, xshift=0.2cm, yshift=0.4cm] 
        at (current page.south west) {
            \includegraphics[width=1.5cm]{Images/logo_ecole.png}
        };
    \end{tikzpicture}
        \end{column}
        
        % --- COLONNE DROITE (IMAGE) ---
        \begin{column}{0.5\textwidth}
            \centering
            \includegraphics[width=\textwidth]{Images/market_share.png}
        \end{column}
        
    \end{columns}
\end{frame}

%-----------------------------------------------
% INTERFACE DE L'OUTIL
%-----------------------------------------------
\section{Interface de l'outil}

\begin{frame}{À quoi est ce que ça ressemble ?}

    % --- 1. L'IMAGE (HAUT) ---
    \vspace{-0.8cm}
    \begin{center}
        % height=0.6\textheight : L'image prend 60% de la hauteur disponible
        % keepaspectratio : Empêche l'image d'être déformée
        \includegraphics[height=0.6\textheight, keepaspectratio]{Images/exemple_trello.png}
    \end{center}
    % --- 2. LA LISTE HORIZONTALE (BAS) ---
        \vspace{-0.7cm}
        \begin{columns}[t] % [t] aligne le texte en haut si un item a plusieurs lignes
        
        \begin{column}{0.33\textwidth}
             \begin{itemize}
                \item Structuration des étapes de la recherche (Kanban)
             \end{itemize}
        \end{column}
        
        \begin{column}{0.33\textwidth}
             \begin{itemize}
                \item Importation de fichers et\\« Power-Ups » 
             \end{itemize}
        \end{column}
        
        \begin{column}{0.33\textwidth}
             \begin{itemize}
                \item Gestion d'équipe
             \end{itemize}
        \end{column}
        
    \end{columns}

    % --- LE LOGO (Toujours en bas à gauche) ---
    \begin{tikzpicture}[remember picture, overlay]
        \node[anchor=south west, xshift=0.2cm, yshift=0.4cm] 
        at (current page.south west) {
            \includegraphics[width=1.5cm]{Images/logo_ecole.png}
        };
    \end{tikzpicture}

\end{frame}

\section {}
\begin{frame}
\centering
\LARGE\textmd{Pour conclure... }
\begin{tikzpicture}[remember picture, overlay]
        \node[anchor=south west, xshift=0.2cm, yshift=0.4cm] 
        at (current page.south west) {
            \includegraphics[width=1.5cm]{Images/logo_ecole.png}
        };
    \end{tikzpicture}
\end{frame}
%-----------------------------------------------
\end{document} 